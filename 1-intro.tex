\section{总体设计}

\subsection{背景与动机}
传统命令式语言(C、Java、Python 等)以赋值为核心,
一旦语句执行完毕,变量之间的语义关系就被切断;
当上游数据变化时,下游值不会自动保持一致。
随着 UI 状态管理、IoT 数据流、实时监控等场景的复杂化,
开发者不得不在回调、订阅和手动更新的迷宫中穿梭,
既容易出现故障,又充满样板代码。

随着现代软件对实时性要求的提高
(如高频交易、即时大盘、复杂的前端交互),
这种"手动同步状态"的开发模式导致了大量的冗余代码和难以维护的回调困境。

现有的响应式库(如 RxJS)提供了补丁式的 API,
但需要工程师自己拼接依赖图、处理订阅生命周期,
无法在编译期验证拓扑是否安全,也缺乏"计算一次即稳定"的强保证。
Ripple 的诞生正是为了解决这一同步危机:
让"变量就是流"成为语言级别的内建语义,
把依赖图视为运行时的一等公民,
用编译期检查和拓扑驱动的执行模型消除中间错误状态。

\subsection{设计目标与核心特性}
\begin{itemize}
  \item \textbf{原生响应式:}
        以 \verb|source|/\verb|stream|/\verb|sink| 三类节点表达依赖关系,
        使用 \verb|<-| 建立持久绑定,避免显式订阅或手动更新。
  \item \textbf{零故障传播:}
        编译期预计算依赖、运行时按拓扑高度调度,
        保证每次输入事件中每个节点最多计算一次,外部永远看到一致状态。
  \item \textbf{静态类型与流类型:}
        基础类型(\verb|int|, \verb|float|, \verb|bool|, \verb|string|)
        与 \verb|Stream<T>| 组合,
        声明式的类型注解有助于早期发现类型与引用错误。
  \item \textbf{声明式语法:}
        类似电子表格的"所见即所得"写法(如 \verb|stream D <- B + C;|),
        强调表达式而非执行顺序。
  \item \textbf{内建时间语义:}
        提供 \verb|pre|(历史值)与 \verb|fold|(状态累积)这样的时序原语,
        无需外部库即可编写有状态流。
  \item \textbf{严格错误检测:}
        编译阶段即捕获循环依赖、未定义引用、重复定义等问题,
        遵循"早失败"原则,避免故障流入运行时。
  \item \textbf{高性能图引擎:}
        运行时采用图归约引擎和基于高度的优先队列,
        兼顾稳定性与事件传播性能,为后续并行和内存局部性优化奠定基础。
\end{itemize}

\subsection{参考与借鉴}
Ripple 的语法与运行时设计参考了多种成熟语言的范式:
\begin{itemize}
  \item \textbf{电子表格模型(Excel):}
        单元格间的持久依赖关系启发了"变量即流"的核心语义。
  \item \textbf{函数式响应式编程(Elm/RxJS):}
        事件流与组合子的思想影响了 \verb|stream| 节点与流式表达式设计,
        但 Ripple 将其下沉到语言与编译器层,并提供拓扑级的安全保证。
  \item \textbf{强类型语言家族(ML/Rust/TypeScript):}
        使用显式类型标注(\verb|name : type|)
        和 \verb|if ... then ... else ...| 风格的表达式语法,
        使流计算同时保持可读性与可验证性。
  \item \textbf{同步数据流语言(Lustre 等):}
        时序原语 \verb|pre| 与 \verb|fold| 借鉴了同步流语言的时间抽象,
        使状态演化自然地融入依赖图。
\end{itemize}
