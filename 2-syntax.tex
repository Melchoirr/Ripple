\section{语法设计}

\subsection{词法设计(Lexical Analysis)}
\paragraph{关键字与保留字} 词法阶段将以下字符串识别为关键字,禁止被当作标识符使用:
$$
\{
\texttt{stream, source, sink, lift, if, then, else, end, match, with, on,}
$$
$$
\texttt{func, let, in, type, pre, fold, map, filter, reduce, int, float,}
$$
$$
\texttt{bool, string, Stream, true, false}
\}
$$

\paragraph{标识符} 使用正则 \verb|[A-Za-z_][A-Za-z0-9_]*|。若命中关键字表则归类为对应关键字 Token,否则归类为 \verb|IDENTIFIER|。

\paragraph{数字字面量} 支持十进制整数与浮点:
\begin{itemize}
  \item 整数:\verb|[0-9]+|,产出 \verb|INT_LITERAL|。
  \item 浮点:\verb|[0-9]+\.[0-9]+|(仅单个小数点),产出 \verb|FLOAT_LITERAL|。
\end{itemize}

\paragraph{字符串字面量} 双引号包围,支持常见转义:\verb|"([^"\\]|\\.)*"|,产出 \verb|STRING_LITERAL|。

\paragraph{布尔字面量} \verb|true|/\verb|false| 直接映射为 \verb|BOOL_LITERAL|。

\paragraph{运算符与分隔符} 由最长优先规则匹配:
\begin{itemize}
  \item 二字符及以上:\verb|<-|(流绑定)、\verb|:=|(源初始化)、\verb|~>|(流向)、\verb|==|、\verb|!=|、\verb|<=|、\verb|>=|、\verb|&&|、\verb||||、\verb|=>|(Lambda)、\verb|=|(函数定义)、\verb|<-| 后的 \verb|on| 触发器。
  \item 单字符:\verb|+ - * / % < > ! ( ) { } [ ] , ; : .|.
\end{itemize}

\paragraph{注释与空白} 单行注释以 \verb|//| 开始至行尾。空格、制表符、回车被跳过。换行仅用于行列计数,不参与语法(\verb|NEWLINE| 仅保留定位信息)。

\paragraph{Token 序列} 词法器输出 Token 序列并在结尾追加 \verb|EOF|,供递归下降解析器消费。

\subsection{语法规则(Grammar, EBNF)}
Ripple 采用自顶向下递归下降解析,按运算符优先级拆分子产生式。下述 EBNF 与 \verb|ripple_parser.py| 完全一致(从最低到最高优先级列出)。

\paragraph{顶层与声明}
\begin{verbatim}
Program        ::= { Statement } EOF ;
Statement      ::= SourceDecl | StreamDecl | SinkDecl
                 | FuncDecl | TypeDecl ;

SourceDecl     ::= "source" Identifier ":" TypeSignature
                   [ ":=" Expression ] ";"
                 | "source" Identifier ":=" Expression ";" ;

StreamDecl     ::= "stream" Identifier "<-" Expression
                   [ "on" Identifier { "." Identifier } ] ";" ;

SinkDecl       ::= "sink"   Identifier "<-" Expression ";" ;

FuncDecl       ::= "func" Identifier "(" [ Identifier 
                    { "," Identifier } ] ")" "=" Expression ";" ;

TypeDecl       ::= "type" Identifier "=" StructType ";" ;
\end{verbatim}

\paragraph{类型系统}
\begin{verbatim}
TypeSignature  ::= BasicType | StreamType | ArrayType
                 | StructType | Identifier ;
BasicType      ::= "int" | "float" | "bool" | "string" ;
StreamType     ::= "Stream" "<" TypeSignature ">" ;
ArrayType      ::= "[" TypeSignature "]" ;
StructType     ::= "{" [ Identifier ":" TypeSignature
                   { "," Identifier ":" TypeSignature } ] "}" ;
\end{verbatim}

\paragraph{表达式(运算符优先级从低到高)}
\begin{verbatim}
Expression     ::= LogicalOr ;
LogicalOr      ::= LogicalAnd { "||" LogicalAnd } ;
LogicalAnd     ::= Equality   { "&&" Equality   } ;
Equality       ::= Comparison { ("==" | "!=") Comparison } ;
Comparison     ::= Additive   { ("<" | ">" | "<=" | ">=") Additive } ;
Additive       ::= Multiplicative { ("+" | "-") Multiplicative } ;
Multiplicative ::= Unary { ("*" | "/" | "%") Unary } ;
Unary          ::= ("!" | "-") Unary | Postfix ;
Postfix        ::= Primary { "[" Expression "]"
                           | "." Identifier } ;
\end{verbatim}

\paragraph{基础与高阶表达式}
\begin{verbatim}
Primary        ::= Literal
                 | Identifier
                 | FunctionCall
                 | IfExpr
                 | LetExpr
                 | PreOp
                 | FoldOp
                 | MapOp
                 | FilterOp
                 | ReduceOp
                 | LambdaExpr
                 | "(" Expression ")" ;

FunctionCall   ::= Identifier "(" [ Expression { "," 

Expression } ] ")" ;
IfExpr         ::= "if" Expression "then" Expression
                   "else" Expression "end" ;
LetExpr        ::= "let" Identifier "=" Expression
                   "in" Expression ;
PreOp          ::= "pre" "(" Identifier "," Expression ")" ;
FoldOp         ::= "fold" "(" Expression "," Expression ","
                   LambdaExpr ")" ;
MapOp          ::= "map" "(" Expression "," LambdaExpr ")" ;
FilterOp       ::= "filter" "(" Expression "," LambdaExpr ")" ;
ReduceOp       ::= "reduce" "(" Expression "," Expression ","
                   LambdaExpr ")" ;
LambdaExpr     ::= "(" [ Identifier { "," Identifier } ] ")" "=>"
                   Expression ;

(* 内置函数 - 通过 FunctionCall 调用 *)
BuiltinFunc    ::= (* 数组操作 *)
                   "len" | "head" | "tail" | "last" | "reverse"
                 | "transpose"
                   (* 聚合函数 *)
                 | "sum" | "avg" | "min" | "max"
                 | "count" | "count_if"
                   (* 数学函数 *)
                 | "abs" | "sqrt"
                   (* CSV 函数 *)
                 | "load_csv" | "col" | "row" | "csv_header" ;

Literal        ::= INT_LITERAL | FLOAT_LITERAL
                 | STRING_LITERAL | BOOL_LITERAL ;
Identifier     ::= see lexical rules ;
\end{verbatim}

\paragraph{完整性说明}
该 EBNF 覆盖所有实现特性:
\begin{itemize}
  \item 带触发器的流声明(\verb|on a.b|)
  \item 数组/结构体类型与字面量、字段访问链
  \item Lambda 形参数量不限
  \item map/filter/reduce 高阶数组操作
  \item let 表达式的局部作用域
  \item pre/fold 时序原语
\end{itemize}

内置函数包括:
\begin{itemize}
  \item 数组操作:\verb|len|、\verb|head|、\verb|tail|、\verb|last|、
        \verb|reverse|、\verb|transpose|
  \item 聚合函数:\verb|sum|、\verb|avg|、\verb|min|、\verb|max|、
        \verb|count|、\verb|count_if|
  \item 数学函数:\verb|abs|、\verb|sqrt|
  \item CSV 函数:\verb|load_csv|、\verb|col|、\verb|row|、\verb|csv_header|
\end{itemize}

解析过程中,\verb|StreamDecl| 还会调用 \verb|extract_dependencies| 和
\verb|is_stateful_expr| 计算静态依赖、拓扑高度与有状态标记,
为调度器提供元数据。

\subsection{抽象语法树(AST)设计}
\paragraph{节点族与元数据} AST 在 \verb|ripple_ast.py| 中分三类:
\begin{itemize}
  \item \textbf{类型节点:} \verb|BasicType|、\verb|StreamType|、
        \verb|ArrayType|、\verb|StructType|、\verb|FunctionType|。
  \item \textbf{表达式节点:} \verb|Literal|、\verb|Identifier|、
        \verb|UnaryOp|、\verb|BinaryOp|、\verb|FunctionCall|、
        \verb|IfExpression|、\verb|LetExpression|、\verb|Lambda|、
        \verb|PreOp|、\verb|FoldOp|、\verb|MapOp|、\verb|FilterOp|、
        \verb|ReduceOp|、\verb|ArrayLiteral|、\verb|ArrayAccess|、
        \verb|StructLiteral|、\verb|FieldAccess|。
  \item \textbf{声明节点:} \verb|SourceDecl|、\verb|StreamDecl|、
        \verb|SinkDecl|、\verb|FuncDecl|、\verb|TypeDecl|,
        程序根节点为 \verb|Program|。
\end{itemize}
其中 \verb|SourceDecl|/\verb|StreamDecl|/\verb|SinkDecl| 携带
\verb|rank|、\verb|is_stateful|、\verb|static_dependencies| 等拓扑元信息,
供运行时调度(\verb|ripple_engine.py|)直接使用。

\paragraph{示例 AST 图} 三段具有代表性的代码,展示 AST 及拓扑元数据。

\textbf{示例 1:无状态菱形依赖}
\begin{verbatim}
source A : int := 1;
stream B <- A * 2;
stream C <- A + 1;
stream D <- B + C;
sink output <- D;
\end{verbatim}
\begin{figure}[h]
\centering
\begin{forest}
for tree={
  grow'=east,
  parent anchor=east,
  child anchor=west,
  draw,
  rounded corners,
  align=center,
  font=\small,
  edge={-Latex},
  s sep=7mm,
  l sep=8mm,
  anchor=center
},
decl/.style={fill=blue!10},
sink/.style={fill=orange!10},
source/.style={fill=green!15},
stream/.style={fill=purple!10},
expr/.style={fill=white}
[Program, decl
  [{SourceDecl\\name=A\\type=int\\rank=0\\stateful=false}, source
    [{Literal\\1:int}, expr]
  ]
  [{StreamDecl\\name=B\\rank=1\\stateful=false}, stream
    [{BinaryOp\\(* )}, expr
      [{Identifier\\A}, expr]
      [{Literal\\2:int}, expr]
    ]
  ]
  [{StreamDecl\\name=C\\rank=1\\stateful=false}, stream
    [{BinaryOp\\(+ )}, expr
      [{Identifier\\A}, expr]
      [{Literal\\1:int}, expr]
    ]
  ]
  [{StreamDecl\\name=D\\rank=2\\stateful=false}, stream
    [{BinaryOp\\(+ )}, expr
      [{Identifier\\B}, expr]
      [{Identifier\\C}, expr]
    ]
  ]
  [{SinkDecl\\name=output}, sink
    [{Identifier\\D}, expr]
  ]
]
\end{forest}
\caption{示例 1:无状态菱形依赖 AST}
\end{figure}

\textbf{示例 2:有状态折叠与条件分支}
\begin{verbatim}
source tick : int := 1;
stream count <- fold(tick, 0, (acc, x) => acc + x);
stream parity <- if count % 2 == 0 then "even" else "odd" end;
sink result <- parity;
\end{verbatim}

\begin{figure}[h]
\centering
\begin{forest}
for tree={
  grow'=east,
  parent anchor=east,
  child anchor=west,
  draw,
  rounded corners,
  align=center,
  font=\small,
  edge={-Latex},
  s sep=7mm,
  l sep=8mm,
  anchor=center
},
decl/.style={fill=blue!10},
sink/.style={fill=orange!10},
source/.style={fill=green!15},
stream/.style={fill=purple!10},
expr/.style={fill=white}
[Program, decl
  [{SourceDecl\\name=tick\\type=int\\rank=0\\stateful=false}, source
    [{Literal\\1:int}, expr]
  ]
  [{StreamDecl\\name=count\\rank=1\\stateful=true}, stream
    [{FoldOp}, expr
      [{Identifier\\tick}, expr]
      [{Literal\\0:int}, expr]
      [{Lambda\\(acc, x) => acc + x}, expr
        [{BinaryOp\\(+)}, expr
          [{Identifier\\acc}, expr]
          [{Identifier\\x}, expr]
        ]
      ]
    ]
  ]
  [{StreamDecl\\name=parity\\rank=2\\stateful=false}, stream
    [{IfExpression}, expr
      [{BinaryOp\\(==)}, expr
        [{BinaryOp\\( \% )}, expr  % 注意这里必须用 \% 转义
          [{Identifier\\count}, expr]
          [{Literal\\2:int}, expr]
        ]
        [{Literal\\0:int}, expr]
      ]
      [{Literal\\"even":string}, expr]
      [{Literal\\"odd":string}, expr]
    ]
  ]
  [{SinkDecl\\name=result}, sink
    [{Identifier\\parity}, expr]
  ]
]
\end{forest}
\caption{示例 2:fold 状态累积 + 条件分支 AST}
\end{figure}

\textbf{示例 3:高阶数组操作与 Lambda(较高复杂度)}
\begin{verbatim}
source numbers : [int] := [1,2,3,4,5];
stream squares <- map(numbers, (x) => x * x);
stream evens <- filter(squares, (x) => x % 2 == 0);
stream total <- reduce(evens, 0, (acc, x) => acc + x);
sink output <- total;
\end{verbatim}

\begin{figure}[htbp]
\centering
\makebox[\textwidth][c]{
\begin{forest}
for tree={
  grow'=east,
  parent anchor=east,
  child anchor=west,
  draw,
  rounded corners,
  align=center,
  font=\small,
  edge={-Latex},
  s sep=7mm,
  l sep=4mm,
  anchor=center
},
decl/.style={fill=blue!10},
sink/.style={fill=orange!10},
source/.style={fill=green!15},
stream/.style={fill=purple!10},
expr/.style={fill=white}
[Program, decl
  [{SourceDecl\\name=numbers\\type=\texttt{[int]}\\rank=0\\stateful=false}, source
    [{ArrayLiteral\\\texttt{[1,2,3,4,5]}}, expr]
  ]
  [{StreamDecl\\name=squares\\rank=1\\stateful=false}, stream
    [{MapOp}, expr
      [{Identifier\\numbers}, expr]
      [{Lambda\\(x) => x * x}, expr
        [{BinaryOp\\(* )}, expr
          [{Identifier\\x}, expr]
          [{Identifier\\x}, expr]
        ]
      ]
    ]
  ]
  [{StreamDecl\\name=evens\\rank=2\\stateful=false}, stream
    [{FilterOp}, expr
      [{Identifier\\squares}, expr]
      [{Lambda\\(x) => \\ x \% 2 == 0}, expr
        [{BinaryOp\\(==)}, expr
          [{BinaryOp\\( \% )}, expr
            [{Identifier\\x}, expr]
            [{Literal\\2:int}, expr]
          ]
          [{Literal\\0:int}, expr]
        ]
      ]
    ]
  ]
  [{StreamDecl\\name=total\\rank=3\\stateful=false}, stream
    [{ReduceOp}, expr
      [{Identifier\\evens}, expr]
      [{Literal\\0:int}, expr]
      [{Lambda\\(acc,x) => acc + x}, expr
        [{BinaryOp\\(+ )}, expr
          [{Identifier\\acc}, expr]
          [{Identifier\\x}, expr]
        ]
      ]
    ]
  ]
  [{SinkDecl\\name=output}, sink
    [{Identifier\\total}, expr]
  ]
]
\end{forest}
}
\caption{示例 3:Map/Filter/Reduce + Lambda AST}
\end{figure}

\paragraph{语义附着} \verb|extract_dependencies| 将字段访问折叠为完整路径(如 \verb|pos.x|),\verb|is_stateful_expr| 递归识别 \verb|pre|/\verb|fold|,使语法树直接携带调度所需的静态信息。这一设计将“语法结构”与“执行拓扑”对齐,确保编译期即可生成无故障传播所需的运行时蓝图。