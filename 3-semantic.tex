\section{语义设计}
本章给出 Ripple 的形式语义。
我们采用"\textbf{静态图语义 + 大步表达式语义 + 小步事件语义}"的分层方式:
编译阶段把声明映射为拓扑有序的依赖图(\verb|ripple_compiler.py|),
表达式在节点内部按大步指称式规则求值,
事件传播由运行时图引擎(\verb|ripple_engine.py|)按照 rank 和触发器进行小步归约。
整个语义链路确保:同一事件步中每个节点至多计算一次,且观察到的状态一致。

\subsection{域的定义(Domains)}

\paragraph{值域}
\[
\mathit{Val} ::= \mathbb{Z} \mid \mathbb{R} \mid \mathbb{B} \mid \mathit{String}
  \mid [\mathit{Val}] \mid \{\mathit{Field}\to\mathit{Val}\} \mid \bot
\]
其中 $\mathbb{B}=\{\texttt{true},\texttt{false}\}$,$\bot$ 表示未初始化值。

\paragraph{环境与存储}
$\rho: \mathit{Id} \to \mathit{Val}$ 为求值环境,$\sigma: \mathit{Id} \to \mathit{Val}$ 为全局存储。

\paragraph{依赖图}
编译期将程序归约为依赖图 $G$,每个节点包含:
\[
\mathit{Node} = (\mathit{deps}, \mathit{trig}, \mathit{rank}, \mathit{stateful}, e)
\]
\begin{itemize}
  \item $\mathit{deps}$:数据依赖集合
  \item $\mathit{trig}$:触发依赖集合(默认 $\mathit{trig}=\mathit{deps}$)
  \item $\mathit{rank}$:拓扑序号,$rank = 1 + \max\{rank(d) \mid d \in deps\}$
  \item $\mathit{stateful}$:是否包含 \verb|pre|/\verb|fold|
\end{itemize}

\paragraph{运行时状态}
每个节点维护:$\mathit{cur}$(当前值)、$\mathit{prev}$(上一时刻值,供 \verb|pre| 使用)、
$\mathit{dirty}$(本轮是否已更新)。

\subsection{辅助函数}
\begin{itemize}
  \item $\textsf{deps}(e)$:提取表达式 $e$ 的静态依赖,对应 \verb|extract_dependencies|
  \item $\textsf{stateful}(e)$:判断是否包含 \verb|pre|/\verb|fold|
  \item $\llbracket e \rrbracket_{\rho}$:在环境 $\rho$ 下求值表达式 $e$
  \item $\mathit{subs}(n)$:节点 $n$ 的下游订阅者集合
\end{itemize}

\subsection{类型系统与类型推导}

类型推导记作 $\Gamma \vdash e : \tau$,表示在类型环境 $\Gamma$ 下表达式 $e$ 具有类型 $\tau$。

\paragraph{类型定义}
\[
\tau ::= \texttt{int} \mid \texttt{float} \mid \texttt{bool} \mid \texttt{string}
  \mid [\tau] \mid \{\ell_1:\tau_1, \ldots\} \mid \texttt{Stream}<\tau> \mid \tau_1 \to \tau_2
\]

\paragraph{基础类型规则}
\begin{align*}
\Gamma \vdash n : \texttt{int} \quad &\text{(整数字面量)} \\
\Gamma \vdash r : \texttt{float} \quad &\text{(浮点字面量)} \\
\Gamma \vdash b : \texttt{bool} \quad &\text{(布尔字面量)} \\
\Gamma \vdash s : \texttt{string} \quad &\text{(字符串字面量)} \\
\Gamma \vdash x : \Gamma(x) \quad &\text{(变量查找)}
\end{align*}

\textbf{示例}:
\begin{verbatim}
source count : int := 0;        // int 类型
source rate : float := 0.05;    // float 类型
source enabled : bool := true;  // bool 类型
source name : string := "test"; // string 类型
\end{verbatim}

\paragraph{运算符类型}
\begin{align*}
\frac{\Gamma \vdash e_1 : \tau_1 \quad \Gamma \vdash e_2 : \tau_2 \quad \tau_1, \tau_2 \in \{\texttt{int}, \texttt{float}\}}{\Gamma \vdash e_1 \odot e_2 : \textsf{promote}(\tau_1, \tau_2)} \quad &\text{(算术运算,$\odot \in \{+,-,*,/,\%\}$)} \\[6pt]
\frac{\Gamma \vdash e_1 : \tau \quad \Gamma \vdash e_2 : \tau}{\Gamma \vdash e_1 \bowtie e_2 : \texttt{bool}} \quad &\text{(比较运算,$\bowtie \in \{==, !=, <, >, \le, \ge\}$)} \\[6pt]
\frac{\Gamma \vdash e_1 : \texttt{bool} \quad \Gamma \vdash e_2 : \texttt{bool}}{\Gamma \vdash e_1 \land e_2 : \texttt{bool}} \quad &\text{(逻辑运算)}
\end{align*}
其中 $\textsf{promote}(\texttt{int}, \texttt{int}) = \texttt{int}$,否则为 $\texttt{float}$。

\textbf{示例}(类型提升):
\begin{verbatim}
source a : int := 10;
source b : float := 2.5;
stream sum <- a + b;      // 结果类型为 float(int 提升为 float)
stream prod <- a * 3;     // 结果类型为 int(int * int = int)
stream cmp <- a > 5;      // 结果类型为 bool(比较运算)
stream logic <- cmp and (b < 10.0);  // bool and bool = bool
\end{verbatim}

\paragraph{复合类型规则}
\begin{align*}
\frac{\Gamma \vdash e_i : \tau \quad \forall i \in 1..n}{\Gamma \vdash [e_1, \ldots, e_n] : [\tau]} \quad &\text{(数组字面量)} \\[6pt]
\frac{\Gamma \vdash e : [\tau] \quad \Gamma \vdash i : \texttt{int}}{\Gamma \vdash e[i] : \tau} \quad &\text{(数组索引)} \\[6pt]
\frac{\Gamma \vdash e_i : \tau_i \quad \forall i \in 1..n}{\Gamma \vdash \{\ell_1: e_1, \ldots, \ell_n: e_n\} : \{\ell_1:\tau_1, \ldots, \ell_n:\tau_n\}} \quad &\text{(结构体字面量)} \\[6pt]
\frac{\Gamma \vdash e : \{\ldots, \ell : \tau, \ldots\}}{\Gamma \vdash e.\ell : \tau} \quad &\text{(字段访问)}
\end{align*}

\textbf{示例}(数组与结构体):
\begin{verbatim}
// 数组:所有元素必须同类型
source nums : [int] := [1, 2, 3, 4, 5];
stream first <- nums[0];              // 类型为 int
stream matrix <- [[1,2], [3,4]];      // 类型为 [[int]]

// 结构体:字段可以是不同类型
type Point = { x: float, y: float };
source origin : Point := { x: 0.0, y: 0.0 };
stream xCoord <- origin.x;            // 类型为 float
\end{verbatim}

\paragraph{控制流类型}
\begin{align*}
\frac{\Gamma \vdash b : \texttt{bool} \quad \Gamma \vdash e_1 : \tau \quad \Gamma \vdash e_2 : \tau}{\Gamma \vdash \texttt{if } b \texttt{ then } e_1 \texttt{ else } e_2 : \tau} \quad &\text{(条件表达式)} \\[6pt]
\frac{\Gamma \vdash e_1 : \tau_1 \quad \Gamma[x \mapsto \tau_1] \vdash e_2 : \tau_2}{\Gamma \vdash \texttt{let } x = e_1 \texttt{ in } e_2 : \tau_2} \quad &\text{(let 绑定)} \\[6pt]
\frac{\Gamma[x_1:\tau_1, \ldots, x_n:\tau_n] \vdash e : \tau}{\Gamma \vdash (x_1, \ldots, x_n) \Rightarrow e : (\tau_1, \ldots, \tau_n) \to \tau} \quad &\text{(Lambda 表达式)}
\end{align*}

\textbf{示例}(控制流):
\begin{verbatim}
source x : int := 10;

// if-then-else:两个分支必须类型一致
stream abs_x <- if x >= 0 then x else -x end;  // 类型为 int

// let 绑定:局部变量引入
stream doubled <- let y = x * 2 in y + 1 end;  // 类型为 int

// Lambda 表达式:匿名函数
stream square <- (n) => n * n;  // 类型为 int -> int
\end{verbatim}

\paragraph{高阶函数类型}
\begin{align*}
\frac{\Gamma \vdash arr : [\tau_1] \quad \Gamma \vdash f : \tau_1 \to \tau_2}{\Gamma \vdash \texttt{map}(arr, f) : [\tau_2]} \\[6pt]
\frac{\Gamma \vdash arr : [\tau] \quad \Gamma \vdash p : \tau \to \texttt{bool}}{\Gamma \vdash \texttt{filter}(arr, p) : [\tau]} \\[6pt]
\frac{\Gamma \vdash arr : [\tau_1] \quad \Gamma \vdash init : \tau_2 \quad \Gamma \vdash f : (\tau_2, \tau_1) \to \tau_2}{\Gamma \vdash \texttt{reduce}(arr, init, f) : \tau_2}
\end{align*}

\textbf{示例}(高阶函数):
\begin{verbatim}
source nums : [int] := [1, 2, 3, 4, 5];

// map: [int] + (int -> float) = [float]
stream halves <- map(nums, (x) => x / 2.0);

// filter: [int] + (int -> bool) = [int]
stream evens <- filter(nums, (x) => x % 2 == 0);

// reduce: [int] + string + (string, int) -> string = string
stream csv <- reduce(nums, "", (acc, x) => acc + "," + x);
\end{verbatim}

\paragraph{时序操作符类型}
\begin{align*}
\frac{\Gamma \vdash x : \tau \quad \Gamma \vdash init : \tau}{\Gamma \vdash \texttt{pre}(x, init) : \tau} \\[6pt]
\frac{\Gamma \vdash s : \tau_1 \quad \Gamma \vdash init : \tau_2 \quad \Gamma \vdash f : (\tau_2, \tau_1) \to \tau_2}{\Gamma \vdash \texttt{fold}(s, init, f) : \tau_2}
\end{align*}

\textbf{示例}(时序操作符):
\begin{verbatim}
source value : int := 0;

// pre: 返回上一时刻的值,初始值类型必须与流类型一致
stream prev_value <- pre(value, -1);  // 类型为 int

// fold: 累积状态,结果类型由初始值和累积函数决定
stream total <- fold(value, 0, (acc, v) => acc + v);  // 类型为 int

// fold 也可用于类型转换
stream history <- fold(value, "", (s, v) => s + " " + v);  // 类型为 string
\end{verbatim}

\paragraph{声明类型检查}
\begin{itemize}
  \item \verb|source x : τ := e|:检查 $\vdash e : \tau'$ 且 $\tau' \le \tau$(子类型兼容)。
  \item \verb|stream x <- e|:推导 $\vdash e : \tau$,将 $x : \texttt{Stream}<\tau>$ 加入环境。
  \item \verb|func f(x₁,...,xₙ) = e|:在 $\Gamma[x_i : \texttt{any}]$ 下推导 $e$ 类型,记录函数签名。
\end{itemize}

\textbf{示例}(声明与类型检查):
\begin{verbatim}
// source: 显式类型注解,初始值必须兼容
source counter : int := 0;           // 正确:0 是 int
source ratio : float := 1;           // 正确:int 可提升为 float
// source bad : int := "hello";      // 错误:string 不兼容 int

// stream: 类型自动推导
stream doubled <- counter * 2;       // 推导为 Stream<int>
stream mixed <- counter + ratio;     // 推导为 Stream<float>

// func: 参数类型在调用时确定
func double(x) = x * 2;
stream result <- double(counter);    // 此处 x 被推断为 int
\end{verbatim}

\subsection{静态语义:从声明到依赖图}

编译阶段将各类声明转换为依赖图节点:

\paragraph{源节点} \verb|source x := e| 编译为:
\[
G[x] = (\varnothing, \varnothing, 0, \bot, e)
\]
源节点无依赖,$rank=0$,初始值为 $\llbracket e \rrbracket$。

\paragraph{流节点} \verb|stream x <- e| 编译为:
\[
G[x] = (\textsf{deps}(e), \textsf{deps}(e), 1+\max_{d}\{rank(d)\}, \textsf{stateful}(e), e)
\]

\paragraph{输出节点} \verb|sink x <- e| 与流节点类似,但不向下游传播。

\paragraph{函数定义} \verb|func f(x) = e| 存入函数环境 $\mathcal{F}[f \mapsto (x, e)]$,供求值时调用。

\subsection{表达式大步语义}

表达式求值记作 $\llbracket e \rrbracket_{\rho} = v$,主要规则如下:

\paragraph{基本表达式}
\begin{align*}
\llbracket c \rrbracket &= c & \text{(字面量)} \\
\llbracket x \rrbracket_{\rho} &= \rho(x) & \text{(变量)} \\
\llbracket e_1 \odot e_2 \rrbracket &= \llbracket e_1 \rrbracket \odot \llbracket e_2 \rrbracket & \text{(二元运算)}
\end{align*}

\paragraph{条件与函数}
\begin{align*}
\llbracket \texttt{if } b \texttt{ then } e_1 \texttt{ else } e_2 \rrbracket &=
\begin{cases}
\llbracket e_1 \rrbracket & b = \texttt{true} \\
\llbracket e_2 \rrbracket & b = \texttt{false}
\end{cases} \\
\llbracket (x) \Rightarrow e \rrbracket_{\rho} &= \langle x, e, \rho \rangle \quad\text{(闭包)} \\
\llbracket f(a_1, \ldots, a_n) \rrbracket_{\rho} &= \llbracket e_f \rrbracket_{\rho[x_1 \mapsto v_1, \ldots, x_n \mapsto v_n]}
\end{align*}
其中 $\mathcal{F}[f] = (x_1, \ldots, x_n, e_f)$,$v_i = \llbracket a_i \rrbracket_{\rho}$。

\paragraph{递归函数}
递归函数调用通过环境中的函数绑定实现。设 $\mathcal{F}$ 为全局函数环境:
\[
\llbracket f(a_1, \ldots, a_n) \rrbracket_{\rho, \mathcal{F}} = \llbracket e_f \rrbracket_{\rho', \mathcal{F}}
\]
其中 $\rho' = \rho[x_1 \mapsto \llbracket a_1 \rrbracket, \ldots, x_n \mapsto \llbracket a_n \rrbracket]$。
递归调用时,函数体 $e_f$ 可再次引用 $f$,因 $\mathcal{F}$ 始终可见。

\textbf{示例}(阶乘):
\begin{verbatim}
func factorial(n) = if n <= 1 then 1 else n * factorial(n - 1) end;
\end{verbatim}
求值 $\llbracket \texttt{factorial}(3) \rrbracket$ 展开为:
\[
3 \times \llbracket \texttt{factorial}(2) \rrbracket = 3 \times 2 \times \llbracket \texttt{factorial}(1) \rrbracket = 3 \times 2 \times 1 = 6
\]

\paragraph{时序操作符}
\begin{align*}
\llbracket \texttt{pre}(x, init) \rrbracket &=
\begin{cases}
\mathit{prev}(x) & \mathit{prev}(x) \neq \bot \\
\llbracket init \rrbracket & \text{否则}
\end{cases} \\
\llbracket \texttt{fold}(s, init, (a,x) \Rightarrow e) \rrbracket &= \llbracket e \rrbracket_{[a \mapsto state, x \mapsto \llbracket s \rrbracket]}
\end{align*}
其中 $state$ 为累积状态(首次为 $init$)。

\paragraph{数组操作}

\subparagraph{数组字面量与索引}
\begin{align*}
\llbracket [e_1, \ldots, e_n] \rrbracket_{\rho} &= [\llbracket e_1 \rrbracket_{\rho}, \ldots, \llbracket e_n \rrbracket_{\rho}] \\
\llbracket arr[i] \rrbracket_{\rho} &= \llbracket arr \rrbracket_{\rho}[\llbracket i \rrbracket_{\rho}] \quad\text{(0-索引)}
\end{align*}

\subparagraph{高阶数组函数}
\begin{align*}
\llbracket \texttt{map}(arr, f) \rrbracket &= [\llbracket f \rrbracket(v) \mid v \in \llbracket arr \rrbracket] \\
\llbracket \texttt{filter}(arr, p) \rrbracket &= [v \mid v \in \llbracket arr \rrbracket, \llbracket p \rrbracket(v) = \texttt{true}] \\
\llbracket \texttt{reduce}(arr, init, f) \rrbracket &= \textsf{foldl}(\llbracket f \rrbracket, \llbracket init \rrbracket, \llbracket arr \rrbracket)
\end{align*}
其中 $\textsf{foldl}(f, z, []) = z$,$\textsf{foldl}(f, z, x::xs) = \textsf{foldl}(f, f(z, x), xs)$。

\subparagraph{内置数组函数}
\begin{align*}
\llbracket \texttt{len}(arr) \rrbracket &= |arr| \\
\llbracket \texttt{head}(arr) \rrbracket &= arr[0] \quad\text{(首元素)} \\
\llbracket \texttt{tail}(arr) \rrbracket &= [arr[1], \ldots, arr[n-1]] \quad\text{(除首元素)} \\
\llbracket \texttt{last}(arr) \rrbracket &= arr[n-1] \quad\text{(末元素)} \\
\llbracket \texttt{sum}(arr) \rrbracket &= \sum_{v \in arr} v \\
\llbracket \texttt{reverse}(arr) \rrbracket &= [arr[n-1], \ldots, arr[0]] \\
\llbracket \texttt{transpose}(M) \rrbracket &= M^T \quad\text{(矩阵转置,$M^T_{ij} = M_{ji}$)}
\end{align*}

\subparagraph{聚合函数}
\begin{align*}
\llbracket \texttt{avg}(arr) \rrbracket &= \frac{\sum_{v \in arr} v}{|arr|} \quad\text{(平均值)} \\
\llbracket \texttt{min}(arr) \rrbracket &= \min_{v \in arr} v \quad\text{(最小值)} \\
\llbracket \texttt{max}(arr) \rrbracket &= \max_{v \in arr} v \quad\text{(最大值)} \\
\llbracket \texttt{count}(arr) \rrbracket &= |arr| \quad\text{(等价于 len)} \\
\llbracket \texttt{count\_if}(arr, p) \rrbracket &= |\{v \in arr \mid \llbracket p \rrbracket(v) = \texttt{true}\}| \quad\text{(条件计数)}
\end{align*}

\subparagraph{CSV 函数}
\begin{align*}
\llbracket \texttt{load\_csv}(path, skip) \rrbracket &= \text{解析 CSV 文件为二维数组,$skip=\texttt{true}$ 时跳过首行} \\
\llbracket \texttt{col}(data, i) \rrbracket &= [row[i] \mid row \in data] \quad\text{(提取第 $i$ 列)} \\
\llbracket \texttt{row}(data, i) \rrbracket &= data[i] \quad\text{(提取第 $i$ 行)} \\
\llbracket \texttt{csv\_header}(path) \rrbracket &= \text{返回 CSV 首行作为字符串数组}
\end{align*}
CSV 函数在编译期求值并缓存,运行时通过文件监听实现热重载。

\subsection{运行时小步语义:事件传播}

事件传播在图引擎中按 rank 递增顺序进行,使用优先队列 $Q$ 调度。

\paragraph{传播算法}
\begin{enumerate}
  \item 源节点收到新值 $v$:更新 $\sigma[x] = v$,将下游 $\mathit{subs}(x)$ 加入队列
  \item 取出 $Q$ 中 rank 最小的节点 $n$
  \item 若触发条件满足,重新计算:$v = \llbracket G(n).e \rrbracket$
  \item 更新状态:$\mathit{cur}(n) = v$,$\mathit{prev}(n) = \mathit{cur}_{old}(n)$
  \item 将 $\mathit{subs}(n)$ 加入队列,重复直到 $Q$ 为空
\end{enumerate}

\paragraph{终止性保证}
由于图无环(编译期检查)且 rank 严格递增,传播必在有限步内终止,每个节点每轮最多计算一次。

\subsection{语义特性与保证}
\begin{itemize}
  \item \textbf{确定性:}
        大步求值与调度顺序均由 rank 全序决定,
        且每个节点在每个时间步最多被计算一次,故同一输入事件的输出唯一。
  \item \textbf{故障免疫:}
        无环性与触发器约束保证不会出现中间态被下游读取的"glitch";
        \verb|pre| 使用 $\mathit{prev}$,\verb|fold| 使用累积状态,
        避免回读未完成的值。
  \item \textbf{时间推进:}
        每次事件结束后 $\Sigma(n).\mathit{prev}$ 设为 $\Sigma(n).\mathit{cur}$,
        为下一时刻的 \verb|pre| 准备历史值。
  \item \textbf{可扩展性:}
        若加入新的控制流(如 \verb|match|)或效应,可在大步语义中增加对应规则;
        若引入并行调度,只需保持 rank 约束与触发条件即可保证结果与顺序执行等价。
\end{itemize}

%\paragraph{小结} 以上语义把“语法 $\rightarrow$ 拓扑 $\rightarrow$ 事件执行”三层清晰分离,又通过 $\textsf{deps}$、rank、stateful 标记将它们联结起来,形式化刻画了 Ripple 的零故障传播与有状态流演化过程。

