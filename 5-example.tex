\section{语言特性展示与代码示例}
本章选取了若干覆盖核心语法语义的代表性程序,逐一进行分析。
所有示例均可直接用 \verb|python ripple_runner.py <file.rpl>| 运行,
或通过 \verb|RippleCompiler().run(code)| 在 Python 中嵌入执行。
示例对应源码位置:\verb|examples/*.rpl|、\verb|test_comprehensive.py|,
可随时交叉验证。

\subsection{基础功能:变量、算术与 I/O 输出}
以下程序验证最小闭环:源定义、算术表达式、Sink 输出与事件传播。
\begin{verbatim}
// 基础功能演示
source a : int := 1;
source b : float := 2.5;

// 四则与比较
stream sum        <- a + b;              // 混合数值 -> float
stream expr       <- (a * 3 + 4) - b;    // 优先级与括号
stream equal_test <- a * 2 == 2 * a;     // 布尔结果

// I/O 输出
sink sum_out   <- sum;
sink expr_out  <- expr;
sink bool_out  <- equal_test;
\end{verbatim}
\paragraph{行为分析}
\begin{itemize}
  \item \textbf{初值求值:}
        编译时源 \verb|a|/\verb|b| 绑定存储,
        Sink 依赖图 rank 分别为 1(\verb|sum|、\verb|expr|、\verb|equal_test|),
        保证自上而下的单次计算。
  \item \textbf{类型推断:}
        \verb|a| 是 \verb|int|,\verb|b| 是 \verb|float|,
        算术结果被提升为 \verb|float|;比较表达式始终返回 \verb|bool|。
  \item \textbf{事件验证:} 运行后执行:
\begin{verbatim}
> a = 4
推送事件: a = 4
当前输出: sum_out=6.5, expr_out= (4*3+4)-2.5 = 13.5, bool_out=true

> b = 10
推送事件: b = 10.0
当前输出: sum_out=14.0, expr_out= (4*3+4)-10 = 6.0, bool_out=true
\end{verbatim}
    每次推送只重算受影响节点,\verb|bool_out| 因仅依赖 \verb|a| 不随 \verb|b| 变化,
    证明触发集正确。
\end{itemize}

\subsection{控制流与函数:条件、递归与"循环"}
Ripple 的控制流以表达式为中心:
\verb|if ... then ... else ... end|、\verb|func|、\verb|let|,
没有命令式 while,而是用递归或 \verb|fold/pre| 完成累积式循环。
本节示例源自 \verb|examples/example7_functions.rpl|。
\begin{verbatim}
// 条件与函数
func clamp(x, lo, hi) = if x < lo then lo
                        else if x > hi then hi else x end end;

func factorial(n) = if n <= 1 then 1 else n * factorial(n - 1) end;

// let 形成局部绑定
func hypot(x, y) =
    let sx = x * x in
    let sy = y * y in
    sqrt(sx + sy);

// 输入
source x : int := 5;
source y : int := 3;

// 调用与递归
stream x_clamped <- clamp(x, 0, 10);
stream fact_x    <- factorial(x);
stream hypot_xy  <- hypot(x, y);

// 通过 pre 实现“事件步迭代计数器”
stream tick <- pre(tick, 0) + 1 on x;  // 仅 x 更新时自增

sink clamp_out <- x_clamped;
sink fact_out  <- fact_x;
sink hypot_out <- hypot_xy;
sink tick_out  <- tick;
\end{verbatim}
\paragraph{行为分析}
\begin{itemize}
  \item \textbf{条件分支:}
        \verb|clamp| 经过双层 \verb|if| 嵌套,
        命中最先满足的分支后短路求值,
        与语义规则 \textsc{IfT}/\textsc{IfF} 一致。
  \item \textbf{递归函数:}
        \verb|factorial| 在求值时按运行时函数环境递归调用,
        不影响静态依赖图(函数体视作纯表达式);
        递归深度由输入决定,可处理中等规模整数。
  \item \textbf{let 作用域:}
        \verb|hypot| 中 \verb|sx|/\verb|sy| 仅在函数体可见,
        验证局部环境遮蔽规则(参见 \verb|_infer_let_expression|)。
  \item \textbf{累积式"循环":}
        \verb|tick| 使用 \verb|pre| 读取前一事件步的自身值并加一,
        仅在 \verb|x| 变化时触发,体现事件驱动式循环;
        初次求值时 \verb|pre| 取初值 0。
  \item \textbf{事件序列验证:}
\begin{verbatim}
初始: x=5 -> clamp_out=5, fact_out=120, hypot_out=5.83, tick_out=1
> x = 2   => clamp_out=2,  fact_out=2,   tick_out=2  (仅 x 触发 tick)
> y = 4   => clamp_out=2,  fact_out=2,   tick_out=2  (y 不触发 tick)
\end{verbatim}
    说明触发器与递归函数均能在拓扑调度下稳定运行,无重复计算。
\end{itemize}

\subsection{高级特性:数组、结构体与状态原语}
\paragraph{数组与高阶操作(map/filter/reduce)}
代码改编自 \verb|examples/example3_array.rpl|,
覆盖数组字面量、索引、内置函数与高阶 Lambda。
\begin{verbatim}
source x : int := 2;
stream arr        <- [x, x * 2, x * 3];
stream doubled    <- map(arr,   (n) => n * 2);
stream evens      <- filter(arr,(n) => n % 2 == 0);
stream reduced    <- reduce(arr, 0, (acc, n) => acc + n);
stream filtered_sum <- reduce(filter(arr, (n) => n > 3),
                              0, (acc, n) => acc + n);

sink arr_out    <- arr;
sink doubled_out<- doubled;
sink evens_out  <- evens;
sink sum_out    <- reduced;
sink fsum_out   <- filtered_sum;
\end{verbatim}
当 \verb|x=3| 时:
\begin{verbatim}
arr = [3, 6, 9]
doubled = [6, 12, 18]
evens = [6]
sum_out = 18
fsum_out = 15   // 过滤出 >3 的 6 和 9 后再 reduce
\end{verbatim}
拓扑调度保证 \verb|arr| 更新一次,\verb|map/filter/reduce| 各自只遍历一次,
且 \verb|filter| 的中间数组不影响其他节点。

\paragraph{结构体与字段级触发}
代码来自 \verb|examples/example8_structs.rpl|,
验证类型别名、字段展开与按字段触发。
\begin{verbatim}
type Point = { x: int, y: int };
source p : Point := { x: 3, y: 4 };

// 字段访问与派生
stream px <- p.x;
stream py <- p.y;
func square(n) = n * n;
stream distance <- sqrt(square(p.x) + square(p.y));

// 组合新的结构体
stream doubled <- { x: p.x * 2, y: p.y * 2 };

sink px_out <- px;
sink py_out <- py;
sink dist_out <- distance;
sink doubled_out <- doubled;
\end{verbatim}
\paragraph{行为分析}
\begin{itemize}
  \item \textbf{类型展开:}
        编译期将 \verb|Point| 展开,
        \verb|p.x|/\verb|p.y| 作为独立源节点参与 rank 计算,便于字段级推送。
  \item \textbf{字段触发:}
        执行 \verb|engine.push_event("p.x", 6)| 仅使依赖 \verb|p.x| 的节点入队,
        \verb|py_out| 保持 4,\verb|dist_out| 更新为 $\sqrt{36+16}$。
  \item \textbf{整体更新:}
        \verb|engine.push_event("p", {x:0,y:0})| 会批量推送字段节点,
        所有依赖随之更新,验证虚拟节点到字段节点的展开逻辑。
\end{itemize}

\paragraph{fold + 结构体累积}
选自 \verb|test_comprehensive.py| 的第三组测试,证明 fold 可以累积结构体状态。
\begin{verbatim}
type Stats = { count: int, sum: int };
source data : [int] := [1, 2, 3, 4, 5];

stream stats <- fold(data, { count: 0, sum: 0 }, (acc, x) => {
    count: acc.count + 1,
    sum:   acc.sum + x
});

stream count   <- stats.count;
stream average <- stats.sum / stats.count;
sink count_out <- count;
sink avg_out   <- average;
\end{verbatim}
初始输出 \verb|count_out=5|、\verb|avg_out=3.0|,推送 \verb|data = [2, 4]| 后,fold 重新累积得到 \verb|count_out=2|、\verb|avg_out=3.0|,说明 fold 的累积状态与数组源变化一致,且结构体字段可直接投影为新流。

\subsection{综合案例:多指标股票信号(复杂逻辑跑通证明)}
完整代码见 \verb|examples/example6_complex.rpl|。特性覆盖:\verb|fold| 累积、\verb|pre| 历史值、条件分支、字符串比较、多 Sink 仪表板。
\begin{verbatim}
source price : float := 100.0;
stream price_sum   <- fold(price, 0.0, (acc, p) => acc + p);
stream price_count <- fold(price, 0.0, (acc, p) => acc + 1.0);
stream sma         <- price_sum / price_count;

stream prev_price      <- pre(price, 100.0);
stream price_change    <- price - prev_price;
stream price_change_pct<- if prev_price > 0.0
                          then (price_change / prev_price) * 100.0
                          else 0.0 end;

stream trend <- if price > sma then "uptrend"
                else if price < sma then "downtrend"
                else "neutral" end end;

stream volatility <- if price_change_pct > 5.0 || price_change_pct < -5.0
                     then "high"
                     else if price_change_pct > 2.0 || price_change_pct < -2.0
                     then "medium" else "low" end end;

stream signal <- if trend == "uptrend" && volatility != "high"
                 && price_change_pct > 1.0 then "BUY"
                 else if trend == "downtrend" && volatility != "high"
                 && price_change_pct < -1.0 then "SELL"
                 else "HOLD" end end;
\end{verbatim}
\paragraph{事件序列与正确性}
\begin{itemize}
  \item \textbf{t=0,price=100:} \verb|sma=100|,\verb|trend=neutral|,\verb|volatility=low|,\verb|signal=HOLD|。fold/prev 初值生效。
  \item \textbf{t=1,price=102:} \verb|price_change_pct=2%|,\verb|trend=uptrend|,\verb|volatility=medium|(2\% 阈值),\verb|signal=BUY|。证明链式条件与字符串比较正确执行。
  \item \textbf{t=2,price=95:} \verb|price_change_pct \approx -6.86%|,\verb|trend=downtrend|,\verb|volatility=high|,\verb|signal=HOLD|(高波动阻止卖出)。显示多条件防抖策略有效。
  \item \textbf{单次计算保证:} 每个时间步 \verb|price_change|、\verb|trend|、\verb|signal| 均只计算一次,因 rank 拓扑和触发器确保无故障传播。
\end{itemize}

\subsection{CSV 数据处理}
本节展示 Ripple 对 CSV 文件的读取、列提取、过滤与聚合能力。示例代码位于 \verb|examples/example9_csv.rpl|。
\begin{verbatim}
// CSV 数据处理示例
source data := load_csv("examples/test_data.csv", true);

// 基础统计
stream row_count <- len(data);
stream salaries <- col(data, 2);           // 提取第3列(薪资)
stream avg_salary <- avg(salaries);
stream total_salary <- sum(salaries);

// 过滤:筛选 active=true 的行
stream active <- filter(data, (row) => row[3]);
stream active_count <- len(active);

// 转换:提取所有姓名
stream names <- map(data, (row) => row[0]);

sink rows_out <- row_count;
sink avg_out <- avg_salary;
sink total_out <- total_salary;
sink active_out <- active_count;
sink names_out <- names;
\end{verbatim}
\paragraph{行为分析}
\begin{itemize}
  \item \textbf{自动类型推断:} \verb|load_csv| 解析 CSV 时自动将单元格转为 \verb|int|/\verb|float|/\verb|bool|/\verb|string|,无需手动声明。
  \item \textbf{列提取:} \verb|col(data, 2)| 返回所有行的第 3 列(0 索引),类型为 \verb|[float]|。
  \item \textbf{聚合函数:} \verb|avg|、\verb|sum| 对数值数组求平均和总和。
  \item \textbf{热重载:} 修改 CSV 文件后,依赖图自动重新计算,输出实时更新。
\end{itemize}

\subsection{多 CSV 数据分析}
本节展示从多个 CSV 数据源构建业务分析仪表板。示例代码位于 \verb|examples/example10_multi_csv.rpl|。
\begin{verbatim}
// 多数据源加载
source employees := load_csv("examples/test_data.csv", true);
source products := load_csv("examples/products.csv", true);
source orders := load_csv("examples/orders.csv", true);

// 员工分析
stream emp_count <- len(employees);
stream emp_salaries <- col(employees, 2);
stream avg_salary <- avg(emp_salaries);
stream total_payroll <- sum(emp_salaries);

// 产品分析
stream prod_prices <- col(products, 2);
stream prod_stocks <- col(products, 3);
stream avg_price <- avg(prod_prices);
stream total_inventory <- sum(prod_stocks);

// 订单分析
stream order_quantities <- col(orders, 2);
stream total_items_sold <- sum(order_quantities);

// 跨数据源计算
stream estimated_revenue <- total_items_sold * avg_price;
stream orders_per_employee <- len(orders) / emp_count;

// 仪表板输出
sink emp_total <- emp_count;
sink emp_avg_salary <- avg_salary;
sink prod_avg_price <- avg_price;
sink biz_revenue <- estimated_revenue;
\end{verbatim}
\paragraph{行为分析}
\begin{itemize}
  \item \textbf{多源依赖:} 三个 CSV 源独立加载,各自的下游节点形成独立子图。
  \item \textbf{跨源计算:} \verb|estimated_revenue| 依赖 \verb|orders| 和 \verb|products| 两个源,任一变化都会触发重算。
  \item \textbf{仪表板模式:} 多个 \verb|sink| 输出形成实时仪表板,每次数据变更后显示最新指标。
  \item \textbf{并行监听:} 三个 CSV 文件同时被监听,任一文件修改都会触发对应分支的更新传播。
\end{itemize}

%\paragraph{小结} 以上示例覆盖了 Ripple 的主要语言面向:基础算术与输出、表达式式控制流、函数与递归、fold/pre 状态循环、数组高阶操作、结构体与字段触发、CSV 数据处理与多源分析,以及复杂业务组合。所有代码均可直接运行或由测试脚本验证,证明语言在拓扑调度与类型检查的支撑下可以稳健地跑通复杂逻辑。