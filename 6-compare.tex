\section{相关语言对比分析}
本章站在"响应式/数据流"大类视角,对比 Ripple 与主流语言或工具在
范式、时间/调度语义、类型安全、故障避免、执行效率、工具链等方面的差异。
选取的基线既包含学术界的同步数据流语言,
也涵盖工业界流行的库与低门槛工具,
以便从多个维度刻画 Ripple 的定位与取舍。

\subsection{对比基线与选择理由}
\begin{itemize}
  \item \textbf{RxJS/ReactiveX(JavaScript 库)}:
        前端最常用的响应式库,代表"库式 FRP",
        对比可见 Ripple 的"语言级"优势与语法简洁性差异。
  \item \textbf{Elm(静态类型前端语言)}:
        现代 FRP-ish 语言,具备类型系统与信号模型,
        可对比 Ripple 的运行时调度与可扩展性。
  \item \textbf{Lustre/Signal(同步数据流语言家族)}:
        实时/控制系统的黄金标准,强调时间同步与形式化验证,
        是 Ripple 时间语义的主要灵感来源。
  \item \textbf{电子表格(Excel/Google Sheets)}:
        极低门槛的"公式即依赖"范式,
        可对比 Ripple 的抽象与可验证性。
  \item \textbf{TypeScript + MobX/Redux Toolkit(状态管理框架)}:
        工程界常见的状态/数据流组合方案,
        用于对比样板代码量、可维护性与性能。
\end{itemize}

\subsection{核心差异比较}
{\setlength{\tabcolsep}{4pt}\scriptsize
\begin{tabularx}{\linewidth}{p{1.0cm}p{2.0cm}p{2.2cm}p{1.6cm}p{2.0cm}p{1.6cm}p{2.4cm}}
\toprule
\textbf{维度} & \textbf{Ripple} & \textbf{RxJS} & \textbf{Elm} &
  \textbf{Lustre} & \textbf{电子表格} & \textbf{TS+MobX} \\
\midrule
范式与载体
  & 原生响应式语言,依赖图+表达式;Python实现便于嵌入。
  & JS库式FRP,Observable链式组合子。
  & 函数式前端语言,TEA架构。
  & 同步数据流语言,嵌入式/安全关键。
  & 单元格=公式的数据流工具。
  & 命令式语言+状态管理框架。 \\
时间/调度
  & 同步事件步,rank+触发器一次求值。
  & 异步推流,调度器用户选。
  & 逻辑时间=消息循环,运行时驱动。
  & 同步时钟,每tick静态调度。
  & 事件驱动重算,依赖隐式。
  & 事件循环+观察者,middleware控制异步。 \\
类型检查
  & 显式类型+Stream$\langle$T$\rangle$;环/重复/引用检查。
  & 依赖TS/JS,类型安全取决于使用。
  & 强类型+推断+穷尽检查,无any。
  & 强类型,时钟类型系统。
  & 动态/弱类型,运行时报\#VALUE!。
  & TS类型;store/selector约束手写。 \\
状态/历史
  & pre、fold原语;节点驻留历史/累积。
  & scan等组合子,无语言级历史。
  & 状态在Model,历史需手存。
  & pre/延迟/窗口完备且静态安全。
  & 手写上一行/辅助格,无fold/pre。
  & 状态在store,历史自行保存。 \\
故障一致性
  & 无环+rank调度防glitch;触发过滤。
  & 顺序/订阅时机不慎易glitch。
  & 单向数据流较安全,复杂依赖需拆分。
  & 因果性检查+拓扑保证无glitch。
  & 菱形依赖常出旧值,依赖引擎优化。
  & 依赖selector/memo,受更新顺序影响。 \\
执行效率
  & 解释执行;触发器减重算;可迁移高性能后端。
  & 事件驱动高效但订阅/分配多;回压手管。
  & 编译JS体积小;不可变带复制成本。
  & 编译C,硬实时友好,占用低。
  & 粗粒度重算,大表易慢。
  & Redux拷贝多,MobX跟踪有开销。 \\
工具链
  & AST/DOT可视化;形式语义;测试覆盖核心特性。
  & Marble测试/浏览器调试,无静态验证。
  & 友好编译错误;可用Elm Review。
  & 工业级验证(model checking、抽象解释)。
  & 原生图表强,无形式化/静态检查。
  & Redux DevTools时间线;无形式化保证。 \\
简洁扩展
  & 语法接近数学式;无需手写订阅;易迁移后端。
  & 组合子多但链式样板重。
  & 语法简洁,框架约束强,偏前端。
  & 专业语法,曲线高,领域窄。
  & 门槛低,复杂逻辑靠技巧,扩展弱。
  & 生态大但样板多,跨组件依赖易复杂。 \\
\bottomrule
\end{tabularx}}
\normalsize

\subsection{简洁性与表达力比较}
\paragraph{语言级 vs. 库级}
Ripple 将"流 = 变量"的语义下沉到语法和编译器,\verb|<-| 绑定即依赖,
避免 RxJS/MobX 式的显式订阅、解订阅与生命周期管理;
与电子表格相比,Ripple 继承了"公式即关系"的直观性,
但提供显式类型与结构体/函数,使复杂逻辑不依赖"隐藏单元格技巧"。
Elm 在语法上简洁,但 TEA 消息循环让长链依赖需要样板状态拆分;
Lustre/Signal 的语法专业化,表意清晰但学习成本高。
总体上,Ripple 在"学习成本/表达力"比上介于电子表格(低成本)
与 Lustre(高成本)之间。

\paragraph{控制流与高阶组合}
Ripple 支持 \verb|if|、\verb|match|(语义中描述)、lambda、高阶数组操作,
使数据流与普通表达式无缝组合;
RxJS 的高阶组合子丰富但链式 API 冗长;
Redux/MobX 需要在 reducer/selector 中手写循环或 map,嵌套性差;
Lustre/Signal 也有 map/fold,但通常为固定频率流。
Ripple 的表达式式控制流让"数据流 + 计算"统一在一个语法域内,
减少上下文切换。

\subsection{执行效率与调度策略比较}
\paragraph{调度颗粒度}
Ripple 采用 rank + 触发器的小顶堆,确保每事件步每节点只算一次;
Lustre/Signal 的静态调度开销更低(编译期展开),适合硬实时;
RxJS/MobX 以运行时订阅链条为主,
若依赖关系频繁变化或 selector 未 memo 化,会产生重复计算;
电子表格重算通常基于增量 DAG,但缺乏字段级触发,粒度更粗。
Ripple 在 Python 解释层面牺牲部分原始性能,
但通过字段级触发和状态缓存减少浪费,适合中等规模低延迟场景。

\paragraph{资源与内存}
Ripple 的节点缓存与累积器驻留内存,与 Redux/MobX 类似;
Lustre/Signal 编译后占用最小;
RxJS 频繁分配 Observable/订阅对象,GC 压力相对高。
Ripple 以 Python GC 为主,无额外手写分配器,
工程复杂度低但难以达到 C/LLVM 级别性能;
未来可将"依赖图 + 公式"后端换成 C/LLVM 以缩小差距。

\subsection{一致性、安全与可验证性比较}
\paragraph{glitch 免疫}
Ripple 的静态无环 + rank 调度 + 触发过滤保证菱形依赖无 glitch,
与 Lustre/Signal 的因果性检查同属强保证;
Elm/Redux 基于单向数据流也较安全,但缺少字段级触发,
复杂依赖链可能重复计算;
RxJS/MobX/电子表格在缺乏严格拓扑或因订阅顺序容易出现"旧值被读"现象,
需要工程约定。Ripple 的优势是"编译期发现 + 运行时策略"二重保险。

\paragraph{类型与形式化}
Ripple 提供静态类型检查、结构体展开与流类型,与 Elm、Lustre 类似;
但类型推断弱于 Elm(存在 any 回退),形式化验证工具链尚在起步;
Lustre/Signal 具备成熟的模型检查与抽象解释工具,遥遥领先。
电子表格/RxJS/MobX 类型安全依赖宿主语言或运行时断言,
缺少形式化语义支撑。
Ripple 的完整语义章节为后续验证提供了基础,比库式方案更易扩展验证器。

\subsection{工具链与生态比较}
\paragraph{调试与可视化}
Ripple 自带 AST/DOT 输出与运行时调试接口;
Redux/MobX 有成熟的时间旅行调试;RxJS 有 marble 测试;
Elm 有友好的编译错误提示;Lustre/Signal 生态提供工业级验证工具。
Ripple 目前在 IDE/调试器生态上较弱,
但由于依赖图结构清晰,可快速对接图可视化或时序分析工具。

\paragraph{可扩展性与迁移成本}
Ripple 的"图 + 公式"后端接口明确,可向 C/LLVM/并行调度迁移;
语法保持小而全,便于教学和原型。
Elm/Redux/MobX 拥有庞大社区与库支持,迁移成本低但框架耦合高;
Lustre/Signal 针对特定领域,通用性受限。
对于需要可验证响应式行为且希望低样板的场景
(教学、原型、IoT/前端逻辑),Ripple 具备独特优势;
若追求极致性能或工业认证,目前 Lustre/Signal 更成熟。

\paragraph{结论性定位}
综合来看,Ripple 在"表达简洁性/一致性保证/教学友好"维度有优势,
在"极致性能/生态工具链"维度弱于工业成熟方案。
其长表述目标是:以最小语法负担提供接近同步数据流语言的时间与一致性保障,
并保留向高性能后端迁移的空间。